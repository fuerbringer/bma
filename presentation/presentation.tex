\documentclass[professionalfont,serif,german]{beamer}
\usepackage[utf8]{inputenc}
\usepackage[german]{babel}
\usepackage{xcolor}
\usetheme{Pittsburgh}
%\usetheme{Boadilla}

\usepackage{newtxtext,newtxmath}
\definecolor{pfblue}{RGB}{1, 0, 103}
\definecolor{pfred}{RGB}{255, 0, 86}
\definecolor{pfgreen}{RGB}{0, 255, 0}

\title[Pathfinder-Vergleicher]{Pathfinding-Algorithmen}
\subtitle{Einführung und Vergleich mittels einer Webapplikation}
\author[A. Stoop, S. Fürbringer]{Adrian Stoop und Severin Fürbringer}
\institute[BMZ, EVT18a]{Berufsmaturitätsschule Zürich \\ \tiny{Technik,
Architektur, Life Sciences}}
\date{5. März 2019}

%%%%%%%%%%%%%%%%%%%%%%%%%%%%%%%%%%%%%%%%%%%%%%%%%%%%%%%%%%%%%%%%%%%%%%%%%%%%%%%%

\begin{document}

\frame{\titlepage}

%%%%%%%%%%%%%%%%%%%%%%%%%%%%%%%%%%%%%%%%%%%%%%%%%%%%%%%%%%%%%%%%%%%%%%%%%%%%%%%%

\begin{frame}
  \section[Einführung in Algorithmen und Pathfinder]{Einführung}
  \frametitle{Einführung}
  \framesubtitle{Was ist ein Algorithmus?}
  \begin{columns}[c] % the "c" option specifies center vertical alignment
    \begin{column}[T]{.5\textwidth} % column designated by a command
      \begin{itemize}
        \item \textbf{Plan zur Lösung eines Problems}
        \item Name abstammend von al-Chwarizmi (* 780, lt. Algorismi)
          % (lt. == latein) 
          % Al-Chwarizmi (Iraner)gilt als einer der bedeutendsten Mathematiker,
          % da er sich mit Algebra als elementarer Untersuchungsform
          % beschäftigte.  Er entwickelte sozusagen Algorithmen zur grafischen
          % lösung von linearen und quadratischen Gleichungen
        \item Kochrezepte, Lösungsverfahren für lin. oder quad. Gleichungen
        \item Informatikrelevant (Abläufe sind nachvollziehbar)
      \end{itemize}
    \end{column}
    \begin{column}[T]{.5\textwidth}
      \begin{figure}
        \includegraphics[height=4.5cm]{img/algorismi.png}
        \caption{Statue von al-Chwarizmi in Iran, Quelle: M. Tomczak, 2013}
      \end{figure}
    \end{column}
  \end{columns}
  % Evtl. hier Interaktion mit Publikum "Was ist euer Algorithmus um sich in
  % einer fremden Stadt zu orientieren?"
\end{frame}
\begin{frame}
  \frametitle{Einführung}
  \framesubtitle{Was sind Pathfinding-Algorithmen?}
  \begin{columns}
    \begin{column}[T]{.4\textwidth}
      \begin{itemize}
        \item \textbf{Finden den Weg von \texttt{A} nach \texttt{B}}
        \item Es gibt verschiedene Arten von Pathfinder
        \item Zentrale Rolle in dieser BMA
      \end{itemize}
    \end{column}
    \begin{column}[T]{.6\textwidth}
      \begin{figure}
        \includegraphics[height=3cm]{img/bms.png}
        \caption{BestFirstFinder findet den Weg (Grün ist Start, Rot ist Ende)}
      \end{figure}
    \end{column}
  \end{columns}
\end{frame}

\begin{frame}
  \frametitle{Einführung}
  \framesubtitle{Was hat das mit Mobilität zu tun?}
    Pathfinding-Algorithmen kommen vor in:
    \begin{columns}
      \begin{column}[T]{0.5\textwidth}
        \begin{itemize}
          \item Selbstfahrenden Fahrzeugen
          \item Digitalen Maps (Routenplanung)
          \item Netzwerktechnik
          \item Videospielen
        \end{itemize}
      \end{column}
      \begin{column}[T]{0.5\textwidth}
        \begin{figure}
          \includegraphics[height=3cm]{img/routing.png}
          \caption{Graph eines Computernetzwerks. Quelle: Wikibooks, 2008 (Public Domain)}
        \end{figure}
      \end{column}
    \end{columns}
\end{frame}

\begin{frame}
  \frametitle{Einführung}
  \framesubtitle{Auswahl der Pathfinding-Algorithmen}
  \begin{itemize}
    \item \textcolor{pfblue}{A*}: Der \textbf{intelligenteste} Pathfinder (sprich ``A-Star'')
    \item \textcolor{pfred}{BestFirstFinder}: Der \textbf{``faulste''} Pathfinder
    \item \textcolor{pfgreen}{BreadthFirstFinder}: Der in der \textbf{Breite suchende} Pathfinder (``breadth'' $\rightarrow$ dt. ``Breite'')
  \end{itemize}
\end{frame}

\begin{frame}
  \frametitle{Einführung}
  \framesubtitle{Heuristik}
  Bedeutet \textbf{mit begrenztem Wissen} nach der Lösung suchen.
  \begin{figure}
    \centering
    \includegraphics[height=4.5cm]{img/mathegrafix.jpg}
  \end{figure}
  \begin{equation*}
    d\big((x_1,y_1),(x_2,y_2)\big) = |x_1 - x_{2}| + |y_{1} - y_{2}|
  \end{equation*}
\end{frame}

\begin{frame}
  \frametitle{Ziel}
  \framesubtitle{Was war das Ziel?}
  \begin{itemize}
    \item Programmierung Webapplikation
    \item Einbindung bestehendes Produkt PathFinding.js
    \item Räume automatisch generieren mit Hindernissen
    \item Gleichzeitige Ausführung und Vergleich der Pathfinder
    \item Ausführungsdaten der Pathfinder sammeln
    \item Daten statistisch auswerten
  \end{itemize}
\end{frame}

\begin{frame}
  \frametitle{Konzept und Realisierung}
  \framesubtitle{Struktur der Webapplikation}
  Willkommenspage $\rightarrow$ Einführungspage $\rightarrow$ Visualisierer (weitere Einführung) $\rightarrow$ Pathfinding-Algorithmen-Vergleicher
\end{frame}

\begin{frame}
  \frametitle{Konzept}
  \framesubtitle{Einführungspage}
  \begin{figure}
    \includegraphics[height=7cm]{img/einfuehrung1.png}
  \end{figure}
\end{frame}

\begin{frame}
  \frametitle{Konzept}
  \framesubtitle{Einführung mit Einzeldemos der Pathfinder}
  \begin{figure}
    \includegraphics[height=7cm]{img/visualisierung1.png}
  \end{figure}
\end{frame}

\begin{frame}
  \frametitle{Konzept}
  \framesubtitle{Pathfinding-Algorithmen-Vergleicher}
  \begin{figure}
    \includegraphics[height=7cm]{img/konzept1.png}
  \end{figure}
\end{frame}

\begin{frame}
  \section{Produkt}
  \frametitle{Das Produkt}
  \framesubtitle{Raster}
    \begin{columns}
      \begin{column}[T]{0.5\textwidth}
        Pathfinding-Algorithmen brauchen einen Raum.
      \end{column}
      \begin{column}[T]{0.5\textwidth}
        \begin{figure}
          \includegraphics[height=4cm]{img/grid1.png}
        \end{figure}
      \end{column}
    \end{columns}
\end{frame}

\begin{frame}
  \frametitle{Das Produkt}
  \framesubtitle{Korridore und Wände}
  \begin{columns}
    \begin{column}[T]{0.5\textwidth}
      \begin{itemize}
        \item Leer $\rightarrow$ Korridor (passierbar)
        \item Grau $\rightarrow$ Wand
        \item Blau $\rightarrow$ Startpunkt
        \item Grün $\rightarrow$ Ziel
      \end{itemize}
    \end{column}
    \begin{column}[T]{0.5\textwidth}
      \begin{figure}
        \includegraphics[height=4cm]{img/grid2.png}
      \end{figure}
    \end{column}
  \end{columns}
\end{frame}

\begin{frame}
  \frametitle{Das Produkt}
  \framesubtitle{Pathfinding-Algorithmen werden angewendet}
  \begin{columns}
    \begin{column}[T]{0.5\textwidth}
      \begin{itemize}
        \item Ausführung der \textbf{drei ausgewählten Pathfinding-Algorithmen}.
        \item \textcolor{pfblue}{A*}
        \item \textcolor{pfred}{BestFirstFinder}
        \item \textcolor{pfgreen}{BreadthFirstFinder}
      \end{itemize}
    \end{column}
    \begin{column}[T]{0.5\textwidth}
      \begin{figure}
        \includegraphics[height=4cm]{img/grid3.png}
      \end{figure}
    \end{column}
  \end{columns}
\end{frame}

\begin{frame}
  \section{Realisierung}
  \frametitle{Realisierung}
  \framesubtitle{Realisierung und Infrastruktur}
  \begin{figure}
    \includegraphics[height=4cm]{img/diagram1.png}
  \end{figure}
\end{frame}

\begin{frame}
  \frametitle{Realisierung}
  \framesubtitle{Technische Kernkomponenten}
  Entwicklung neuer* Algorithmen.
  \begin{itemize}
    \item Dynamische Generierung des Rasters (mit Hindernissen)
    \item Funktionen für Ausgabe der Statistikwerte
    \item UI: Grafische Darstellung des Rasters
    \item UI: Steuerung Parameter
    \item Zusammenführung bestehendes Produkt (3 Pathfinder)
  \end{itemize}
  \scriptsize{* Die 3 Pathfinder-Implementationen vom bestehenden Produkt übernommen und etwas erweitert.}
\end{frame}

\begin{frame}
  \section[Vorführung]{Vorführung}
  \frametitle{Vorführung}
  \begin{center}
    URL: \textbf{https://bma.fuerbringer.info}
  \end{center}
\end{frame}


\begin{frame}
  \section[Statistik]{Statistik}
  \frametitle{Statistische Auswertungen}
  \begin{columns}
    \begin{column}[T]{1.1\textwidth}
      \begin{figure}
        \includegraphics[height=7.75cm]{../thesis/images/statistik_all2.JPG}
      \end{figure}
    \end{column}
  \end{columns}
\end{frame}

% https://www.youtube.com/watch?v=YU5ZERVkewM

\begin{frame}
  \section[Vielen Dank für eure Aufmerksamkeit!]{Schluss}
  \framesubtitle{Vielen Dank für eure Aufmerksamkeit!}
  \frametitle{Schluss}
  \begin{itemize}
    \item BMA-Produkt als Webapplikation:\\ \url{bma.fuerbringer.info}
    \item Quelltext Webapplikation und Dokument:\\
      \url{github.com/fuerbringer/bma}
  \end{itemize}
\end{frame}

\end{document}
