\documentclass[professionalfont,serif,german]{beamer}
\usepackage[utf8]{inputenc}
\usepackage[german]{babel}
%\usetheme{Pittsburgh}
\usetheme{Boadilla}

\usepackage{newtxtext,newtxmath}

\title[Pathfinder-Vergleicher]{Pathfinding-Algorithmen}
\subtitle{Einführung und Vergleich mittels einer Webapplikation}
\author[A. Stoop, S. Fürbringer]{Adrian Stoop, Severin Fürbringer}
\institute[BMZ, EVT18a]{Berufsmaturitätsschule Zürich}
\date{\today}

%%%%%%%%%%%%%%%%%%%%%%%%%%%%%%%%%%%%%%%%%%%%%%%%%%%%%%%%%%%%%%%%%%%%%%%%%%%%%%%%

\begin{document}

\frame{\titlepage}

%%%%%%%%%%%%%%%%%%%%%%%%%%%%%%%%%%%%%%%%%%%%%%%%%%%%%%%%%%%%%%%%%%%%%%%%%%%%%%%%

\begin{frame}
  \frametitle{Inhalt}
  \tableofcontents
\end{frame}

%%%%%%%%%%%%%%%%%%%%%%%%%%%%%%%%%%%%%%%%%%%%%%%%%%%%%%%%%%%%%%%%%%%%%%%%%%%%%%%%

\begin{frame}
  \section[Abc]{Hello}
  Hello
\end{frame}

%%%%%%%%%%%%%%%%%%%%%%%%%%%%%%%%%%%%%%%%%%%%%%%%%%%%%%%%%%%%%%%%%%%%%%%%%%%%%%%%
% 
%       IDEAS
% 
%   1)  Einfach verständliches Inhaltsverzeichnis (Oberflächlich)
% 
%   2)  Was ist ein Algorithmus: Mit dem echten Leben verbundenes Beispiel
%       - Am besten mit Medien und/oder Publikum arbeiten
% 
%   3)  Was ist ein Pathfinder: Vorgehen wie 2)
%       - Bezug zum Oberthema MOBILITÄT
% 
%   4)  Ziel der Arbeit: 
%       - Ziel: Was wollten wir herausfinden; Wieso welche Pathfinder auch
%       erläutern; Hypothese darlegen
%       - Wie: Webapplikation (Pathfinder-Vergleicher)
%       - Wieso: Was macht unsere Webapplikation speziell. Evtl. Alternativen
%       zeigen
% 
%   5)  Eingesetzte Methoden (Webapplikation und statistische Auswertung): 
%       - Funktion der Webapplikation: Wie vergleicht die Webapp die
%       Pathfinder?
%       - Statistik: Wie haben wir die Daten erhoben und Ausgewertet?
%
%   6)  Demo der Webapplikation
%       - Teil evtl. zusammen mit der Klasse machen; Jeder kann die
%       Webapplikation auf seinem Handy aufrufen und kurz testen
%       - Parallel dazu eine Bedienungsanleitung
%
%   7)  Zusammenfassung:
%       [^ Bewertung on JP zur Hilfe nehmen, da mehrere Punkte hierzu
%       kritisiert wurden^]
%       - Resultate der Arbeit und statistischen Auswertung verständlich
%       erklären; Stellungnahme zur Hypothese mit Bezug zu den Resultaten
%       nehmen;
%       - Kurze Wiederholung was das ganze Resultat ist
%       - Erweiterungsmöglichkeiten und Schwachstellen
%       
%   8)  Schluss:
%       - Alle relevanten Links aufzeigen (Webapplikation-URL und
%       GitHub-Repository)
% 
%%%%%%%%%%%%%%%%%%%%%%%%%%%%%%%%%%%%%%%%%%%%%%%%%%%%%%%%%%%%%%%%%%%%%%%%%%%%%%%%

\end{document}
