\renewcommand{\abstractname}{Abstract}
\begin{abstract}
Diese Berufsmaturitätsarbeit behandelt folgende Fragestellung: Was sind die technischen Eigenschaften und Vor- und Nachteile von drei ausgewählten Pathfinding-Algorithmen und wie unterscheiden sie sich? 
Um diese Fragestellung zu beantworten, wird im Rahmen dieser Arbeit eine Webapplikation entwickelt, die den Benutzer in das Thema einführt und Vergleiche zwischen drei Pathfinding-Algorithmen interaktiv, parallel und mehrmals hintereinander ermöglicht. 
Um den Vergleich möglich zu machen, generiert die Webapplikation automatisch Raster mit zufällig platzierten Wänden. Die Bedingungen (Parameter) des Vergleichs können vom Benutzer auch angepasst werden.
Die experimentartige Webapplikation liefert für statistische Zwecke Messwerte, welche dann für die Ableitung der Vor- und Nachteile der drei Pathfinding-Algorithmen miteinbezogen werden.
%Pathfinding-Algorithmen sind Programmabläufe, welche in der Wirtschaft und Theorie in vielen Anwendungen vorkommen, wie zum Beispiel in Videospielen, Simulationen oder der Automobilindustrie. Sie berechnen in einem Raum den Weg zwischen zwei Punkten.
%Im Rahmen dieser Berufsmaturitätsarbeit wird eine Webapplikation entwickelt, die den Benutzer in das Thema der Pathfinding-Algorithmen einführt und einen interaktiven Vergleich von den drei ausgewählten Pathfindern A*-, BestFirst-, und BreadthFirstFinder ermöglicht und dazu Messungen mit drei Merkmalen macht. 
%Der Vergleicher die Merkmale der verglichenen Pathfinder ``zurückgelegte Zeit'', ``Operationen'' und ``Rechenzeit''.
%Die Webapplikation generiert für den Vergleich ein Zufallslabyrinth, markiert Start- und Endpunkte und lässt mehrere Vergleiche in Serie geschaltet zu.
%Mithilfe des Pathfinder-Vergleichers werden im Rahmen der Arbeit auch Statistiken erstellt und ausgewertet und im schriftlichen Teil die Funktionsweisen der Pathfinder erläutert.
\\
  Die Webapplikation ist unter \url{bma.fuerbringer.info} zugänglich. Der dazugehörige Quellcode ist frei unter \url{github.com/fuerbringer/bma} erhältlich.
\end{abstract}
